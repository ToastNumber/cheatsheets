\documentclass[12pt, twocolumn]{article}
\usepackage[utf8]{inputenc}
\usepackage{minted}
\usepackage{csquotes}
\usepackage{xcolor}
\usepackage{mdframed}

\title{\vspace{-4cm}Monoids, Functors, Applicative Functors, Monads}
\author{Kelsey McKenna}
\date{}
\begin{document}
\pagenumbering{gobble}
\maketitle

\section*{Monoid}

\subsection*{Definition}
\begin{minted}{haskell}
class Monoid m where
  mempty  :: m
  mappend :: m -> m -> m

  mconcat :: [m] -> m
  mconcat ms = foldr mappend mempty
\end{minted}

\subsection*{Rules}
\begin{minted}{haskell}
mempty <> x   = x
x <> mempty   = x
(x <> y) <> z = x <> (y <> z)
\end{minted}

\section*{Functor}

\subsection*{Definition}
\begin{minted}{haskell}
class Functor m where
  fmap :: (a -> b) -> m a -> m b
\end{minted}

\subsection*{Rules}
\begin{minted}{haskell}
fmap id = id
fmap (g . f) = fmap g . fmap f
\end{minted}

\newpage

\section*{Applicative Functor}

\subsection*{Definition}
\begin{minted}{haskell}
class ApplicativeFunctor m where
  pure :: a -> m a
  <*>  :: m (a -> b) -> m a -> m b
\end{minted}

\subsection*{Rules}
The only `interesting' rule is:
\begin{minted}{haskell}
fmap f x = pure f <*> x
\end{minted}

\section*{Monad}

\subsection*{Definition}
\begin{minted}{haskell}
class Monad m where
  return :: a -> m a
  (>>=) :: m a -> (a -> m b) -> m b

  (>>) :: m a -> m b -> m b
  m a >> m b = m a >>= \_ -> m b
\end{minted}

\subsection*{Rules}
\begin{minted}{haskell}
return a >>= f = f a
m >>= return = m
(m >>= f) >>= g = 
    m >>= (\x -> f x >>= g)
\end{minted}

\end{document}

